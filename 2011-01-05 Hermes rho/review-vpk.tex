\documentclass{article}
\begin{document}

\section{Note to the Editor}
In the last 10 years, Deeply Virtual Exclusive reactions became a power probe of the partonic structure of the nucleon. It was found that in Bjorken limit and small t the amplitude is the convolution of the perturbative kernel with a new class of non-local quark (or gluon) operators, called Generalized Parton Distributions. The GPDs give access to the complex internal structure of the nucleon, such as correlations between the parton transverse spatial and longitudinal momentum distributions. They provide a unified picture of the nucleon form factors, polarized  and unpolarized parton distributions, and provide access to the contribution of the total parton angular momentum  to the nucleon spin.
  
Hard exclusive production of a mesons of the nucleon are the key processes to access  the GPDs from the experimental observables.  
Detailed experimental study of the vector meson production by HERMES collaboration gives us the opportunity
to better understand the production mechanism and compare experimental data with the prediction of different models including GPDs. It is absolutely necessary step in the attempts to understand the nucleon structure.

The HERMES data was used to determine the ratios of the helicity amplitudes. This method gives
better result for the spin-density matrix elements (SDMEs) determination than in the previous analysis based on the SDME method.

The manuscript is very well written, has the detailed introduction and definitions. 
There is detailed analysis of the systematic errors in the paper.
The comparison with some pQCD theoretical models is very impressive. I am sure that physicists  working in this field will read the article with great interest.

I have some questions concerning the data analysis that have to be addressed. There are some misprints in the paper as well.


\section{Notes to the authors}
\subsection{General remarks}
The manuscript is very well written, has the detailed introduction and definitions. 
The comparison with some pQCD theoretical models is very impressive. 

W dependence of the helicity amplitudes.

The helicity amplitudes are the function of three variables: Q2, t and W (or $x_B$). However the binning was done only in two dimensional (Q2,-t') grid. The acceptance usually depends on all three variables. Does it affect the extraction of the helicity amplitudes? It may change at least the average values  of the kinematical variable <Q2> what is important for the comparison with the theoretical models.  

Radiative corrections.

Radiative corrections (RC) are important in electroproduction processes. These corrections are model dependent because the Born term is unknown. It is not clear what kind of RC were applied and what model was used for the Born term. These corrections affect not only absolute cross section (usually at the level of 10\%) but angular distributions as well. Please provide the description of RC and systematic errors of the extracted helicity amplitudes ratios connected with RC.

Definition of <Q2> and <t>.

You provide the mean values of the kinematic variables for 16 bins in Table 1. It is not clear
what is your definition of the average values of the <Q2> and <t'>. There are different ways to estimate  the <Q2> and <-t'> in your bins: weighted with the  cross section or just the average over your bin. Did you check that the extracted cross section in the point (<Q2>,<-t'>) corresponds to the MC cross section sigma(<Q2>, <-t'>)?  

You have 9 parameters in your fit. How did you check that your fit is stable under the variation of initial values of the parameters? These parameters are not known a priory. Taking into account that the number of the parameters is not small you can easily pick up the local extremums.  

Testing the Extracted Method (page 9)

The resulting amplitude ratios was to be found to be consistent with input amplitudes ratios within statistical uncertainties. However these two results are based on the same statistics. 
So the uncertainty must be much smaller than statistical error. To make sure that you have no systematics you have to do another pass with MC program and look to the amplitude ratio again. As I understand you check  the self-consistency only for several bins. What systematic error did you estimate based on this test?
Did you use the corrected MC program to get the final results?

The t-slope is function of all variables (Q2, W). Does your MC contain the dependence of the slope parameter on these variables? Please estimate the systematic error connected with this slope in case you use constant slope parameter.

Page 16, first column, line 7.

The reference to Table 1 is incorrect. Must be Tab. 4.

There is no reference to the previously published experimental data on the same subject:
S.A. Morrow et al., Eur. Phys. J. A 39, 5-31 (2009), Exclusive $\rho^0$ electroproduction on the proton at CLAS.





\end{document}
