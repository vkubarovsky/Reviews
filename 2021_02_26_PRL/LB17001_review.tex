%\documentclass[11pt]{amsart}
\documentclass[11pt]{article}
\usepackage{geometry}                % See geometry.pdf to learn the layout options. There are lots.
\geometry{letterpaper}                   % ... or a4paper or a5paper or ... 


\title{Review of the manuscript\\
"Constraints on New Physics in the Electron g-2 from a Search for Invisible Decays of
a Scalar, Pseudoscalar, Vector, and Axial Vector"}
\author{NA64 Collaboration}
\date{}                                           % Activate to display a given date or no date
\begin{document}
\maketitle
 


The manuscript  "Constraints on New Physics in the Electron g-2 from a Search for Invisible Decays of
a Scalar, Pseudoscalar, Vector, and Axial Vector" presents the result of the experiments performed by the NA64 collaboration at CERN 
in the years 2016, 2017, and 2018. A search for a new generic X boson was performed in the 100 GeV electron scattering  off nuclei followed by invisible decay. This paper continues the previous publications of the NA64 collaboration published in PRL 118, 011802 (2017), PRD 97, 072002 (2018),  and PRL 123, 121801 (2019). 

The manuscript extends the search for a scalar (S), pseudoscalar (P), vector (V) or an axial vector (A) particle while in the previous publications only vector boson was considered. As a result, the NA64 collaboration placed the new bounds on the S, P, V, and A coupling strengths to electrons, and set constrains on the contributions to the electron anomalous magnetic moment  $\alpha_e$.
These limits 
are an order of magnitude more sensitive  compared to the current accuracy on $\alpha_e$
from $g-2$ experiments and recent determination of the fine structure constant. These results are  significant achievement in the field.

The manuscript is well organized and clearly written. 
The technical quality and scientific rigor of the manuscript are very good and the main conclusion is well supported. 
The references to the literature are adequate. 
The title and abstract of the article are informative and clear. 

There is a misprint in the home Institutions list, the references [10] and [17] are identical:
\begin{itemize}
   \item [10] State Scientific Center of the Russian Federation Institute for High Energy Physics
                 of National Research Center 'Kurchatov Institute' (IHEP),142281 Protvino, Russia
   \item [17] State Scientific Center of the Russian Federation Institute for High Energy Physics
         of National Research Center 'Kurchatov Institute' (IHEP), 142281 Protvino, Russia
\end{itemize}

The theoretical introduction is understandable  for a nonspecialist.
The experimental setup is briefly described. However, the experimental details and trigger descriptions are difficult to understand without schematic view of the setup.
I  recommend to add the picture to the paper or at least give clear reference to the article  with the detailed description of the experiment and with the same notation.

The trigger condition in Eq. 3 looks like event selection criteria ${\overline{ECAL}(\leq E^{th}_{ECAL})} $  demands energy deposition in the calorimeter $E_{ECAL}>E^{th}_{ECAL}$ . I believe that the trigger condition ${\overline{ECAL}(\leq E^{th}_{ECAL})} $ has to be changed to 
${{ECAL}(\leq E^{th}_{ECAL})} $ or ${\overline{ECAL}(> E^{th}_{ECAL})} $.

Also the sentence below Eq.8 

"accepting events with in-time hits in beam-defining counters
$S_i$ and clusters in the PS and ECAL with the
energy exceeding the thresholds $E^{th}_{PS}\simeq 0.3~GeV$ and
$E^{th}_{ECAL}\leq 80~GeV$, respectively."

has to be changed to 

"accepting events with in-time hits in beam-defining counters
$S_i$ and energy in the PS 
 exceeding the threshold $E^{th}_{PS}\simeq 0.3~GeV$, and energy in the ECAL below 
$E^{th}_{ECAL}\simeq 80~GeV$."


\end{document}  