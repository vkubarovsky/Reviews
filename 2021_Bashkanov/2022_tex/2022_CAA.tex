
\documentclass[prc,floatfix,superscriptaddress,letter]{revtex4}

\usepackage{color}
\usepackage[normalem]{ulem}
\input epsf

\setlength{\textwidth}{6.5in}
\setlength{\textheight}{9.6in}
\setlength{\oddsidemargin}{0.0in}
\setlength{\topmargin}{-1.0in}
\usepackage{amsmath}
\usepackage{graphicx}
\usepackage{url}
\usepackage{color}
\usepackage{subfigure}
\usepackage{longtable}
\usepackage{dcolumn}



    % See p.105 of "TeX Unbound" for suggested values.
    % See pp. 199-200 of Lamport's "LaTeX" book for details.
    %   General parameters, for ALL pages:
    \renewcommand{\topfraction}{0.9}	% max fraction of floats at top
    \renewcommand{\bottomfraction}{0.8}	% max fraction of floats at bottom
    %   Parameters for TEXT pages (not float pages):
    \setcounter{topnumber}{2}
    \setcounter{bottomnumber}{2}
    \setcounter{totalnumber}{4}     % 2 may work better
    \setcounter{dbltopnumber}{2}    % for 2-column pages
    \renewcommand{\dbltopfraction}{0.9}	% fit big float above 2-col. text
    \renewcommand{\textfraction}{0.07}	% allow minimal text w. figs
    %   Parameters for FLOAT pages (not text pages):
    \renewcommand{\floatpagefraction}{0.7}	% require fuller float pages
	% N.B.: floatpagefraction MUST be less than topfraction !!
    \renewcommand{\dblfloatpagefraction}{0.7}	% require fuller float pages
    
    \newcommand{\jpsi}{$J/\psi$ }
\begin{document}


\title{Review of the CLAS12 Analysis Proposal "Hexaquarks at CLAS12"  \\
by M. Bashkanov, D.P. Watts and  N. Zachariou (the York group)}
\author{Valery Kubarovsky and Zhiwen Zhao}
\date{\today}
\maketitle

The suggested CLAS12 analysis proposal is devoted to the study of the extremely interesting topic. It was suggested to search for the heavy strange partners of the $d^*(2380)$ particle that has possible interpretation as a hexaquark state. This resonance was seen in the $d\pi^0\pi^0$, $d\pi^+\pi^-$ and $pp\pi^0\pi^-$ channels in the hadro- and photo-production experiments. The York group suggested the measurements that will allow the discovery of all  $d^*(2380)$ multiplet members in the case if $d^*$ belongs to the SU(3) antidecuplet. This is very challenging task and this statement is too strong in our view. The group is going on to study hexaquark candidates with strangeness zero, one, two and three using the run group B data set.

The Committee reviewed the CLAS12 Analysis Proposal "Hexaquarks at CLAS12"  and came to the conclusion that this group demonstrated the ability to analyze CLAS12 data. There is no doubts that the goal of the proposed study is at the cutting edge of the particle spectroscopy.
Based on this conclusion we approve the proposal for the further analysis. 


\end{document}